\documentclass[10pt,a4paper]{report}
\usepackage[utf8]{inputenc}
\usepackage{amsmath}
\usepackage{amsfonts}
\usepackage{amssymb}

\title{SGBD Floues}
\author{Jean-Mathieu CHANTREIN}
\date{}
\begin{document}

\maketitle

\chapter{Prédicats floues}

\paragraph{Rappel :\\}
En logique classique, un \textbf{prédicat} est un terme (une expression) évaluer à vrai ou faux.

\paragraph{Buts :\\}
\begin{itemize}
\item Comparer une variable à une valeur constante(ex: $prixCher$)
\item Comparer $n$ variables (ex: $prixDeVente \approx prixVendu$)
\end{itemize}

\paragraph{Moyen :\\}
Définir chaque prédicat flou par un sous ensemble flou[Zadeh65].


Un \textbf{sous ensemble flou} $A$ de $E$ est défini par une fonction d'appartenance $f_A$ sui associe à chaque element $x$ de $E$, le degré d'appartenance $f_A(X)$ avec lequel il appartient à A(entre 0 et 1):$$f_A:E\rightarrow[O;1]$$

DESSIN

\paragraph{Définitions :\\}

Le \textbf{support} du sous ensemble $A$ de $E$ est:$$support(A)=\{x \in E | f_A(x)<>0 \}$$

La \textbf{hauteur} du sous ensemble $A$ de $E$ est:$$hauteur(A)=sup_{x \in E} f_A(x)$$

Le \textbf{noyau} du sous ensemble $A$ de $E$ est:$$noyau(A)=\{x \in E | f_A(x)=1 \}$$

\chapter{Relations floues}
Une \textbf{requete floue(FQUERY)} est une requete exprimée en SQL, dont la partie condition est étendue pour traiter les prédicats flous définis:\\\\
select $A_1$...$A_n$\\
from table\\
where conditions\\

Le \textbf{résultat d'une requête floue} est une \textbf{relation flou} BD, c'est une relation résultat pour lequel chaque tuple a un degré d'appartenance dans $[0;1]$, décrivant le succès de la requête.

\chapter{Opérateurs}
La \textbf{sélection floue} consiste à appliquer un prédicat flou construit par l’utilisation de l'opérateur unaire (NOT) et/ou des opérateurs binaires (AND/OR) sur les prédicats flous donnés.

\par Il faut définir:
\begin{itemize}
\item$f_{!A}(x)$
\item$f_{A\&B}(x)$
\item$f_{A\|B}(x)$
\end{itemize} 

La \textbf{négation} d'un sous ensemble flou (appelé \textbf{complément} en logique flou)$A$ de $E$ est défini par:$$f_{!A}(x)=1-f_A(x)$$

La \textbf{disjonction} (union) de 2 sous ensembles flou $A,B$ de $E$ est défini par:
$$f_{A\|B}(x)=max(f_A(x),f_B(X))$$

La \textbf{conjonction} (intersection) de 2 sous ensembles flou $A,B$ de $E$ est défini par:
$$f_{A\&B}(x)=min(f_A(x),f_B(X))$$

\paragraph{Remarque: \\}
Pour palier à la rigidité de $f_{A\&B}(x)$, on peut utiliser OWA(Ordered Weighted Averaging). Mais cette possibilité ne fait pas partie de la BDD floue:
$$ F_{A_1\&...\&A_i\&...\&A_n}(x)=w_1f_{A_1}(x)+...+w_if_{A_i}(x)+...+w_nf_{A_n}(x)$$
avec $\sum_{i=1}^{n} w_i=1$ et $\forall i,f_{A_i}(x)\geq f_{A_{i+1}}(x)$

\begin{itemize}
\item si $w_1=1$ alors $w_i=0$ et disjonction floue (max)
\item si $w_n=1$ alors $w_i=0$ et intersection floue (min)
\item si $w_i=\frac{1}{n}$ alors moyenne arithmétique
\end{itemize}

\section{Jointure floue}
select R,S\\
from R,S\\
where RC $\approx$ S.D\\
où $\approx$ est un opérateur de jointure flou fournit par $f \approx (a,b)$ 



\chapter{Modificateurs}
\paragraph{Idée: \\}
Utiliser un opérateur qui permettent de moduler des prédicats.Ex: très,peu...
\paragraph{Définition: \\}
Un \textbf{modificateur linguistique} $m$ est une fonction qui associe à tout sous ensemble flou A de E, un sous ensemble flou $m(A)$ de E par l'intermédiaire d'une fonction $t_m$ tel que:
$\forall x \in E,f_{m(A)}(X)=t_m(f_A(X)) $
\paragraph{Exemple :\\}

\begin{tabular}{|c|c|c|c|c|c|c|}
\hline
\multicolumn{6}{|c|}{VENTE} \\

\hline 
Adresse & Age & Prix & $f_{vieux}$ & $f_{tresVieux}$ & $f_{tresCher}$ \\ 
\hline 
FOCH & 20 & 1 & 0 & 0 & 1\\ 
\hline 
IENA & 100 & 0,8 & 1 & 1 & 0\\ 
\hline 
LAFAYETTE & 90 & 0,9 & 0.8 & 0.64 & 0.25\\ 
\hline 
BELLE-BEILLE & 20 & 0,5 & 0 & 0 & 0\\ 
\hline 
BLEU & 80 & 1 & 0.6 & 0.36 & 1\\ 
\hline 
\end{tabular} 

\begin{verbatim}
Select
from Vente
where tresVieux(Age)

Select
from Vente
where tresCher(Prix)

\end{verbatim}

\paragraph{Limitation de la méthode: \\}
La sémantique de "tres" ou "peu" peut-être différente en fonction des mots choisi, or la définition de "très" est générique.

\part{Bases de données et requêtes floues}

\begin{enumerate}
\item FQUERY (sous ensemble flou) : \\
	\begin{itemize}
	\item des données \textbf{précises}
	\item des systemes relationnels
	\item permet des requetes floues\\
	$\longrightarrow$ \textbf{les résultats avec degré (d'appartenance)}
\end{itemize}

\item FSQL (sous ensemble flou + théorie des possibilités) : \\
	\begin{itemize}
	\item des données \textbf{imprécises}
	\item des systèmes relationnels
	\item permet des requêtes floues\\
	$\longrightarrow$ \textbf{les résultats filtrés}
\end{itemize}

\end{enumerate}

\paragraph{Idées de base:\\}
\begin{itemize}
	\item les données imprécises sont stockées dans des tables classiques
	\item requêtes peuvent utiliser des prédicats flou à définir
	\item degré
\end{itemize}

\paragraph{Plan:\\}
\begin{itemize}
	\item A.Stockage des données imprécises et des prédicats flous
	\item B.Le problème
	\item C.Les requêtes
\end{itemize}

\chapter{Stockage des données imprécises \\}
\begin{itemize}
\item Une colonne contient:
\begin{itemize}
\item \textbf{soit une valeur classique}
\item \textbf{soit une valeur floue numérique}
\end{itemize}
\item Le codage d'une valeur floue numérique se fait sous forme d'un trapèze, en fournissant les quatres valeurs remarquables sur l'abscisse
\begin{itemize}
\item La table trapèze contient une clef associée à un trapèze sur un ensemble de données
\item Une référence au trapèze correspondant est stocké dans la table
\end{itemize}
\item Le codage d'un prédicat flou se fait suivant la meme technique qu'une valeur floue
\end{itemize}

\paragraph{\underline{Exemple:} \\\\}

\begin{tabular}{|c|c|c|}
\hline
\multicolumn{3}{|c|}{STAGE}\\
\hline 
Titre & Indemnité & Durée \\ 
\hline 
dev.Web & 500 (a1) & 6 \\ 
\hline 
analyse & [500,1000](500,500,1000,1000) (a2) & 6 \\ 
\hline 
Prog & fig1(0,500,1000,1500) (a3) & 3 \\ 
\hline 
Doc & 500 (a1) & 3 \\ 
\hline 
BDD & [500,1000](500,500,1000,1000) (a2) & 6 \\ 
\hline
\end{tabular} 

\begin{tabular}{|c|c|c|c|c|}
\hline
\multicolumn{5}{|c|}{TRAPEZE}\\
\hline 
a1 & 500 & 500 & 500 & 500 \\ 
\hline 
a2 & 500 & 500 & 1000 & 1000 \\ 
\hline 
a3 & 0 & 500 & 1000 & 1500 \\ 
\hline 
\end{tabular} 

\chapter{B.Le problème \\}

\begin{enumerate}
\item FQUERY : \\
	\begin{itemize}
	\item prédicat flou: sous ensemble flou $f(x)$
	\item une donnée \textbf{précise} $a$
	\item le résultat
	\begin{itemize}
		\item pour une donnée: $f(a)$ le degré d’appartenance
		\item Pour la requête: des combinaisons de ces degrés
	\end{itemize}		
\end{itemize}

\item FSQL : \\
	\begin{itemize}
	\item prédicat flou: sous ensemble flou $f_A(x)$
	\item une donnée \textbf{imprécise}: un sous ensemble flou $f_{A'}(y)$
	\item le résultat
	\begin{itemize}
		\item pour une donnée est un nombre tenant compte de $f_A$ et $f_{A'} f(a)$: le degré d'appartenance $\rightarrow$ comparer deux sous ensemble flous(prédicat et donnée) 
		\item Pour la requête : des combinaisons de ces degrés
	\end{itemize}		
\end{itemize}
\end{enumerate}

\begin{tabular}{|c|c|c|}
\hline 
Adresse & Age & Prix \\ 
\hline 
JOFFRE & 50 & (0.5;1;$+\infty$;$+\infty$) \\ 
\hline 
IENA & 90 & (1.1;1.1;1.2;1.2) \\ 
\hline 
EIFFEL & 60 & 1.8 \\ 
\hline 
JOFFRE & 95 & (0.8;0.9;1.5;1.6) \\ 
\hline 
\end{tabular} 

\chapter{C.Les requêtes}
\paragraph{Problème\\}

Ayant un ensemble flou $ A $ (décrivant un prédicat: $PrixEleve$), et un ensemble flou $ A' $   (décrivant une donnée: $Prix$), on vaut mesurer la confiance en  $ A' $ est un $ A $ (le prix est un prix élevé: $PrixEleve(Prix)$)

\paragraph{Solution : La théorie des possibilités\\}
\begin{itemize}
\item La mesure de possibilité (Optimisme)\\
$ ps(X est A / Xest A')=sup\{min(f_A(x),f_{A'}(x)) / x \in E\} $
\item la mesure de nécessité (Pessimisme)\\
$ nc(X est A / X est A')=inf \{max(1-f_{A'}(x), f_A(x)) / x \in E\} $
\end{itemize}
N.B.: Le résultat du minimum de deux fonctions floues est une fonction floue.
\begin{tabular}{|c|c|c|c|c|}
\hline 
Adresse & Age & Prix & Ps & Nc \\ 
\hline 
JOFFRE & 50 & (0.8;0.9;1.1;1.2) & 0.5 & 0 \\ 
\hline 
IENA & 90 & (1.1;1.1;1.7;1.7) & 1 & 0.5  \\ 
\hline 
EIFFEL & 60 & 1.8 & 1 & 1  \\ 
\hline 
JOFFRE & 95 & (1.3;1.4;1.5;1.6) & 1 & 1  \\ 
\hline 
\end{tabular} 

avec $f_{prixEleve}:(1;1.2;1.8;2)$


Select col1,...,coln,cdeg(ps|n)
from T1,...,Tm
where condition

\begin{itemize}
\item cdeg(ps|n): rend trié sur degré de ps ou nc
\item condition:faible(indemnité)...prixEleve(Prix)...faible(Indemnité) $>$ NFEQ 0,1 retorne les résultats avec un degré de nécéssité supérieur a 0,1\\
faible(indemnité)$>$FEQ 0,1 retourne les résultats avec un degré de possibilité supérieur à 0,1
\end{itemize}

\section*{TD1 Flou, SGBD}
\subsection*{Exercice 1}
1)Relations floues
\begin{itemize}
\item $f_{J\&K}(x)=min(f_J(x),f_K(x))$
\item $f_{J\|K}(x)=max(f_J(x),f_K(x))$
\end{itemize}

2)Théorie des possibilités
\begin{itemize}
\item $Ps(f_J|f_{[18;18;28;28]})=sup\{min(f_J|f_{[18;18;28;28]})\}=1$
\item $Nc(f_J|f_{[18;18;28;28]})=inf\{max(1-f_{[18;18;28;28]},f_J)\}=0$
\end{itemize}
N.B.:Attention à l'ordre prédicat/données\\

3)Modérateurs linguistique\\
$f_{tres}(x)=x^2 $
\subsection*{Exercice 2}

1)Définitions de fonction floues

$f_{GrandDom}=[0;150;200;+\infty]$\\
$f_{Difficile}=[20;25;+\infty;+\infty]$\\
$f_{Facile}=[0;0;20;25]$\\
$f_{Moyen}=[260;280;320;340]$\\
$f_{MoyenneMontagne}=[1300;1500;1500;1700]$\\

2)Requetes et résultats

a)Stations avec un grand domaine skiable\\\\
select S.Nom\\ 
from Stations S\\
where $f_{GrandDom}$(S.Domaine)\\

\begin{tabular}{|c|c|}
\hline 
Nom & $f_{GrandDom}$ \\ 
\hline 
Courch & 1 \\ 
\hline 
DeuxA & 0.8 \\ 
\hline 
Samo & 0.4 \\ 
\hline 
Serre & 1 \\ 
\hline 
\end{tabular} \\\\

b)Stations avec un grand domaine skiable et beaucoup de pistes difficiles\\\\
select S.Nom\\ 
from Stations S\\
where $f_{GrandDom}$(S.Domaine) \& $f_{Difficile}$(S.PisteRN)\\

\begin{tabular}{|c|c|c|c|}
\hline 
Nom & $f_{GrandDom}$ & $f_{Difficile}$ & Resultat \\ 
\hline 
Courch & 1 & 1 & 1 \\ 
\hline 
DeuxA & 0.8 & 1 & 0.8 \\ 
\hline 
Samo & 0.4 & 1 & 0.4 \\ 
\hline 
Serre & 1 & 1 & 1 \\ 
\hline 
\end{tabular}\\\\

c)Stations à environ 1500m avec beaucoup de piste difficiles\\\\
select S.Nom\\ 
from Stations S\\
where $f_{MoyenneMontagne}$(S.Altitude) \& $f_{Difficile}$(S.PisteRN)\\

\begin{tabular}{|c|c|c|c|}
\hline 
Nom & $f_{MoyenneMontagne}$ & $f_{Difficile}$ & Resultat \\ 
\hline 
Courch & 1 & 1 & 1 \\ 
\hline 
Serre & 1 & 1 & 1 \\ 
\hline 
\end{tabular}\\\\

d)Stations ou l'on peut trouver un hotel à prix moyen\\\\
select S.Nom\\ 
from Stations S natural join Hotels H\\
where $f_{Moyen}$(H.Prix) \\

\begin{tabular}{|c|c|}
\hline 
Nom & $f_{Moyen}$ \\ 
\hline 
Samo & 0.5 \\ 
\hline 
Courch & 1 \\ 
\hline 
Pierre & 1 \\ 
\hline 
\end{tabular}\\

N.B: distinct ajoute une borne sup sur la condition\\

e)Stations avec un hôtel à prix moyen ou un grand domaine skiable\\\\
select S.Nom\\ 
from Stations S natural join Hotels H\\
where $f_{Moyen}$(Prix) $\|$ $f_{GrandDom}$\\


\begin{tabular}{|c|c|c|c|}
\hline 
Nom & $f_{Moyen}$ & $f_{GrandDom}$ & $f_{Moyen \| GrandDom}=max(f_{Moyen},f_{GrandDom})$ \\ 
\hline 
Samo & 0.5 & 0.4 & 0.5 \\
\hline 
Courch & 1 & 1 & 1 \\ 
\hline 
Pierre & 1 & 0 & 1 \\ 
\hline 
DeuxA & 0 & 0.8 & 0.8 \\ 
\hline 
Serre & 0 & 1 & 1 \\ 
\hline 
\end{tabular}\\\\

f)Stations où le prix du forfait est environ la moitié du prix d'une chambre: environ, à définir, permet une tolérance linéaire de 50\\\\
select S.Nom\\ 
from Stations S natural join Hotels H\\
where S.Forfait $\approx$ H.Prix/2 \\
avec $f_{\approx} = |X-Y| \leq 50 $\\

\begin{tabular}{|c|c|c|c|c|c|}
\hline 
Nom & IdStation & Prix/2 & Forfait & $f_{\approx}$ & DISTINCT\\ 
\hline 
Samo & 102 & 125 & 150 & $25\rightarrow0.5$ & NON \\ 
\hline 
Samo & 102 & 165 & 150 & $15\rightarrow0.7$ & OUI \\ 
\hline 
Samo & 102 & 120 & 150 & $30\rightarrow0.4$ & NON \\ 
\hline 
Samo & 102 & 130 & 150 & $20\rightarrow0.6$ & NON \\ 
\hline 
Courch & 100 & 155 & 200 & $45\rightarrow0.1$ & NON \\ 
\hline 
Courch & 100 & 225 & 200 & $25\rightarrow0.5$ & OUI \\ 
\hline 
Pierre & 201 & 135 & 90 & $45\rightarrow0.1$ & OUI \\ 
\hline 
Pierre & 201 & 150 & 90 & $60\rightarrow 0$ & NON \\ 
\hline 
\end{tabular} 

\subsection*{Exercice 3}

N.B.Lors d'un calcul de possibilitée ou necessité, il faut considerer les constantes comme des valeurs floues.

1)Il faut définir la table TRAPEZE qui contiendra les fonctions floues que l'on souhaite modéliser: Mûr, Jeune, Vieux, Elevé, Faible\\

2)\\
Select C.Nom\\
from Cadre C\\
where $f_{moyenSup}(C.Salaire)$\\


\part{Le Data Ware House}
\chapter{A.Le problème}
Les bases de données actuellement utilisées sont essentiellement des BD de production de type OLTP:On Line Transaction Processing:
elles se caractérisent par:
\begin{itemize}
\item Les mises à jour des données
\item lecture / ecriture
\item des échanges utilisateurs/BD faibles
\item orientation des interfaces vers 1 ou 2D
\item taille en Giga , Tera octets
\item données récentes
\end{itemize}

Les bases de données d'aide à la décision émergent, elles sont de type OLAP: On Line Analysis Processing; elles se caractérisent par  :
 \begin{itemize}
 \item analyse des données
 \item lecture
 \item echanges utilisateurs/BD forts
 \item orientation des interfaces 3...n Dimensions
 \item taille de Tera/Peta octets
 \item données historiées
 \end{itemize}
$\rightarrow$ elles permmettent de contenir et de synthétiser les données de l'entreprise en vue de comprendre l'entreprise.

\chapter{B.Data Ware House}
La BDD de l'entreprises est concu par domaine d'activité de l'entreprise.\\\\
L'entrepot de données (Data Ware House) regroupe l'ensemble des données historicisées,consolidée dans une unique BDD,afin de permettre de faire de l'aide à la prise de décision.\\\\


A partir de l'entrepot , on peut :
\begin{itemize}
\item faire de l'analyse de données(OLAP): extraire des données de synthèse, essentiellement numérique.
\item faire de la fouille de données(Data Mining) qui extrait des connaissances.
\end{itemize}

\chapter{C.Une architecture}

\section*{La modélisation multidimensionelle(OLAP)}
\begin{itemize}
\item A.Le cube de Données
\begin{itemize}
\item Un cube de données est une représentation visuelle en 3D de données: il est composé de 3 axes représentant des attributs Dimi, et d'un indicateur IndX qui est la variable analysée dans le cube
\item La variable analysée peut etre une agragation
\end{itemize}
\item B.L'exploitation multidimensionnelle
\begin{itemize}
\item Granularité
\begin{itemize}
\item  Chaque dimension peut être granulaire et ainsi plus ou moins développée.
\end{itemize}
\item L'exploitation
\begin{itemize}
\item Le dépliage (Drilldown) est l'extension d'une dimension vers une dimension à grain plus fin.
\item Le pliage(Rollup) est la réduction d'une dimension vers une dimension à grain plus gros
\item La coupe (Slice) est la sélection de tranches du cube selon une dimension
\end{itemize}
\end{itemize}
\end{itemize}

ACHAT
\begin{tabular}{|c|c|c|c|c|}
\hline 
NumFour & NomProd & Date & Quantité & Prix \\ 
\hline 
• & • & • & • & • \\ 
\hline 
\end{tabular} 








\end{document}