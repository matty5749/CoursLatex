\documentclass[a4paper,11pt]{article}

\usepackage[frenchb]{babel}
\usepackage[utf8]{inputenc}
\usepackage[T1]{fontenc}
\usepackage{amsmath}
\usepackage[bookmarks=true,colorlinks,linkcolor=blue]{hyperref}


\usepackage[babel]{csquotes}
\MakeAutoQuote{«}{»}

\usepackage[top=2cm, bottom=2cm, left=2cm, right=2cm]{geometry}

\usepackage{url}
\usepackage{wrapfig}
\usepackage{tabularx}
\usepackage{graphicx}
\usepackage{tikz}
\usepackage{pgfplots}
\usepackage{tikz-3dplot}

\usepackage{color}
\usepackage{verbatim}
\usepackage{listings}
\usepackage{fancyhdr}
\usepackage{subfig}

\definecolor{colKeys}{rgb}{0,0,1} 
\definecolor{colIdentifier}{rgb}{0,0,0} 
\definecolor{colComments}{rgb}{0,0.5,1} 
\definecolor{colString}{rgb}{0.6,0.1,0.1} 

\lstset{%configuration de listings 
float=hbp,% 
basicstyle=\ttfamily\small, % 
identifierstyle=\color{colIdentifier}, % 
keywordstyle=\color{colKeys}, % 
stringstyle=\color{colString}, % 
commentstyle=\color{colComments}, % 
columns=flexible, % 
tabsize=2, % 
frame=l, % 
frameround=tttt, % 
extendedchars=true, % 
showspaces=false, % 
showstringspaces=false, % 
%numbers=left, % 
numberstyle=\tiny, % 
breaklines=true, % 
breakautoindent=true, % 
captionpos=b,%
} 
\usetikzlibrary{positioning}

\lstset{language=SQL} 
\lstset{commentstyle=\textit} 

% Matière - Formation Lieu 
% Cours/TP/TD - Titre
% Auteur
% Date
\title{M2 ID - Bases de données avancées \\ \normalsize (Cours)}
\author{Quentin \textsc{Beilleau} \\ Jean-Mathieu \textsc{Chantrein} \\ David \textsc{Guinehut}}

\date{\today}

\begin{document}

%En-tête
	\lhead{Q. \textsc{Beilleau}, J-M. \textsc{Chantrein},\\ D. \textsc{Guinehut}}
	\rhead{Bases de données avancées \emph{Cours} - M2ID S1 - 2013}
	\renewcommand{\headrulewidth}{0.001pt}
	
	\pagestyle{fancy}
	\fancypagestyle{plain}
	
	\maketitle

	\tableofcontents

\section*{Prélude}
Cours présenté par Stéphane \textsc{Loiseau}, Enseignant-Chercheur au LERIA \emph{(Laboratoire d’Étude et de Recherche en Informatique d'Angers)}.\\

Contact:
	\begin{itemize}
		\item[] \url{stephane.loiseau@univ-angers.fr} 
	\end{itemize}

\part{Base de données Floues, données précises, FQUERY}

\subsection*{Rappel}
En logique classique, un \textbf{prédicat} est un terme (une expression) évalué à vrai ou faux.

\section{Prédicats floues}

\subsection{Buts}
\begin{itemize}
\item Comparer une variable à une valeur constante(ex: $prixCher$)
\item Comparer $n$ variables (ex: $prixDeVente \approx prixVendu$)
\end{itemize}

\subsection{Moyen}
Définir chaque prédicat flou par un sous ensemble flou[Zadeh65].

Un \textbf{sous ensemble flou} $A$ de $E$ est défini par une fonction d'appartenance $f_A$ qui associe à chaque élément $x$ de $E$, le degré d'appartenance $f_A(X)$ avec lequel il appartient à A(entre 0 et 1):$$f_A:E\rightarrow[O;1]$$

		\begin{figure}[!h]
			\centering
			 
\begin{tikzpicture}
	\begin{axis}[
		xlabel={$E$},
		ylabel={$f_A(X)$},
		axis x line=bottom,
		axis y line=left,
		xscale=1.2,
		yscale=0.5,
		xtick=data,
		legend entries={$f_{etreGrand}$,hauteur,noyeau,support},
		legend style={at={(1,1)},anchor=south west}]
		
			\addplot+[mark=none,line width= 2] coordinates{(0,0)(180,0)(200,1)(250,1)};
			\addplot+[mark=none,line width= 4,red] coordinates{(220,0)(220,1)};
			\addplot+[mark=none,line width= 4,green] coordinates{(200,1.04)(250,1.04)};
			\addplot+[mark=none,line width= 4,cyan] coordinates{(180,0.02)(250,0.02)};
			
	\end{axis}	
\end{tikzpicture}
			\caption{CAPTION - TODO}
		\end{figure}
		
		\begin{figure}[!h]
			\centering
			\begin{tikzpicture}
	\begin{axis}[
		xmin=0,
		ymin=0,
		xlabel={$E$},
		ylabel={$f_A(X)$},
		axis x line=bottom,
		axis y line=left,
		xscale=1.2,
		yscale=0.5,
		xtick=data,
		legend entries={$f_{animal}$},
		legend style={at={(1,1)},anchor=south west},
		xticklabels={Chien,Chat,Homme,Microbe}]
		
			\addplot+[only marks,green,line width=3] coordinates{(1,1)(2,1)(3,1)(4,0.7)}; 
	\end{axis}	
\end{tikzpicture}		 

			\caption{CAPTION - TODO}
		\end{figure}
		
\subsection{Définitions}

Le \textbf{support} du sous ensemble $A$ de $E$ est: \begin{center} \fbox{$support(A)=\{x \in E | f_A(x)<>0 \}$} \end{center}

La \textbf{hauteur} du sous ensemble $A$ de $E$ est: \begin{center} \fbox{$hauteur(A)=sup_{x \in E} f_A(x)$} \end{center}

Le \textbf{noyau} du sous ensemble $A$ de $E$ est: \begin{center} \fbox{$noyau(A)=\{x \in E | f_A(x)=1 \}$} \end{center}

\section{Relations floues}
Une \textbf{requete floue(FQUERY)} est une requete exprimée en SQL, dont la partie condition est étendue pour traiter les prédicats flous définis:\\

\begin{lstlisting}[mathescape]
SELECT $A_1$...$A_n$
FROM tableA
WHERE conditions
\end{lstlisting}
~

Le \textbf{résultat d'une requête floue} est une \textbf{relation flou} BD, c'est une relation résultat pour lequel chaque tuple a un degré d'appartenance dans $[0;1]$, décrivant le succès de la requête.

\section{Opérateurs}
La \textbf{sélection floue} consiste à appliquer un prédicat flou construit par l’utilisation de l'opérateur unaire (NOT) et/ou des opérateurs binaires (AND/OR) sur les prédicats flous donnés.\\

Il faut définir:
\begin{itemize}
	\item$f_{!A}(x)$	
	\item$f_{A\&B}(x)$
	\item$f_{A\|B}(x)$
\end{itemize} 
~

La \textbf{négation} d'un sous ensemble flou (appelé \textbf{complément} en logique flou)$A$ de $E$ est défini par: \begin{center} \fbox{$f_{!A}(x)=1-f_A(x)$} \end{center}

La \textbf{disjonction} (union) de 2 sous ensembles flou $A,B$ de $E$ est défini par: \begin{center} \fbox{$f_{A\|B}(x)=max(f_A(x),f_B(X))$} \end{center}

La \textbf{conjonction} (intersection) de 2 sous ensembles flou $A,B$ de $E$ est défini par: \begin{center} \fbox{$f_{A\&B}(x)=min(f_A(x),f_B(X))$} \end{center}

\paragraph{\emph{Remarque:}}
Pour palier à la rigidité de $f_{A\&B}(x)$, on peut utiliser OWA(Ordered Weighted Averaging). Mais cette possibilité ne fait pas partie de la BDD floue:
$$ F_{A_1\&...\&A_i\&...\&A_n}(x)=w_1f_{A_1}(x)+...+w_if_{A_i}(x)+...+w_nf_{A_n}(x)$$
\\

avec $\sum_{i=1}^{n} w_i=1$ et $\forall i,f_{A_i}(x)\geq f_{A_{i+1}}(x)$

\begin{itemize}
	\item si $w_1=1$ alors $w_i=0$ et disjonction floue (max)
	\item si $w_n=1$ alors $w_i=0$ et intersection floue (min)
	\item si $w_i=\frac{1}{n}$ alors moyenne arithmétique
\end{itemize}


\subsection{Jointure floue}
 
\begin{lstlisting}[mathescape]
SELECT R,S
FROM R,S
WHERE RC $\approx$ S.D
\end{lstlisting}
où $\approx$ est un opérateur de jointure flou fournit par $f \approx (a,b)$ 


\section{Modificateurs}
\subsection{Idée}
Utiliser un opérateur qui permettent de moduler des prédicats. Ex: très, peu...
\subsection{Définition}
Un \textbf{modificateur linguistique} $m$ est une fonction qui associe à tout sous ensemble flou A de E, un sous ensemble flou $m(A)$ de E par l'intermédiaire d'une fonction $t_m$ tel que:
$\forall x \in E,f_{m(A)}(x)=t_m(f_A(x)) $

\paragraph{Exemple:}
\begin{center}
		\begin{figure}[!h]
			\centering
			\begin{tikzpicture}
	\begin{axis}[
		xmax=200,
		ymin=0,
		ymax=1,
		xlabel={$E$},
		axis x line=bottom,
		axis y line=left,
		xscale=1.2,
		yscale=0.5,
		xtick=data,
		legend entries={$f_{vieux}$,$f_{tresVieux}$},
		legend style={at={(1,1)},anchor=south west}]
		
			\addplot+[mark=none,line width= 4,cyan] coordinates{(0,0)(50,0)(100,1)(200,1)};
			\addplot+[mark=none,line width= 2,red] coordinates{(0,0)(50,0)};
			\addplot+[domain=50:100,mark=none,line width= 2,red] {1/2500*x^2-1/25*x+1};
			\addplot+[mark=none,line width= 2,red] coordinates{(100,1)(200,1)};
			
	\end{axis}	
\end{tikzpicture} 

			\caption{}{On a defini  $f_{vieux}$ et $tres(x)=x^2$. Cela nous permet d'utiliser le moderateur linguistique $ tres$ sur $f_{vieux}$. On a donc $f_{tresVieux}(x)=tres(f_{vieux}(x))=f_{vieux}^2(x)$}
		\end{figure}				
\end{center}
\emph{\textbf{Remarque : }Il s'agit bien de $f^2(x)$ et non pas $f(x^2)$.}
\begin{center}
\begin{tabular}{|c|c|c|c|c|c|c|}
	\hline
	\multicolumn{6}{|c|}{\textbf{VENTE}} \\
	\hline
	\hline 
	Adresse & Age & Prix & $f_{vieux}$ & $f_{tresVieux}$ & $f_{tresCher}$ \\ 
	\hline 
	\hline
	FOCH & 20 & 1 & 0 & 0 & 1\\ 
	\hline 
	IENA & 100 & 0,8 & 1 & 1 & 0\\ 
	\hline 
	LAFAYETTE & 90 & 0,9 & 0.8 & 0.64 & 0.25\\ 
	\hline 
	BELLE-BEILLE & 20 & 0,5 & 0 & 0 & 0\\ 
	\hline 
	BLEU & 80 & 1 & 0.6 & 0.36 & 1\\ 
	\hline 
\end{tabular} 
\end{center}
\begin{lstlisting}[mathescape]
SELECT
FROM Vente
WHERE tresVieux(Age)

SELECT
FROM Vente
WHERE tresCher(Prix)

\end{lstlisting}

\subsection{Limitation de la méthode}
La sémantique de "tres" ou "peu" peut-être différente en fonction des mots choisi, or la définition de "très" est générique.

\part{Base de données Floues, données imprécises, FSQL}

\begin{enumerate}
\item FQUERY (sous ensemble flou) : \\
	\begin{itemize}
	\item des données \textbf{précises}
	\item des systemes relationnels
	\item permet des requetes floues\\
	$\longrightarrow$ \textbf{les résultats avec degré (d'appartenance)}
\end{itemize}

\item FSQL (sous ensemble flou + théorie des possibilités) : \\
	\begin{itemize}
	\item des données \textbf{imprécises}
	\item des systèmes relationnels
	\item permet des requêtes floues\\
	$\longrightarrow$ \textbf{les résultats filtrés}
\end{itemize}

\end{enumerate}

\subsection{Idées de base}
\begin{itemize}
	\item les données imprécises sont stockées dans des tables classiques
	\item requêtes peuvent utiliser des prédicats flou à définir
	\item degré
\end{itemize}


\section{Stockage des données imprécises}
\begin{itemize}
\item Une colonne contient:
\begin{itemize}
\item \textbf{soit une valeur classique}
\item \textbf{soit une valeur floue numérique}
\end{itemize}
\item Le codage d'une valeur floue numérique se fait sous forme d'un trapèze, en fournissant les quatres valeurs remarquables sur l'abscisse
\begin{itemize}
\item La table trapèze contient une clef associée à un trapèze sur un ensemble de données
\item Une référence au trapèze correspondant est stocké dans la table
\end{itemize}
\item Le codage d'un prédicat flou se fait suivant la meme technique qu'une valeur floue
\end{itemize}

\textbf{Exemple:}
\begin{center}
\begin{tabular}{|c|c|c|}
	\hline
	\multicolumn{3}{|c|}{\textbf{STAGE}}\\
	\hline 
	\hline
	Titre & Indemnité & Durée \\ 
	\hline
	\hline 
	dev.Web & 500 (a1) & 6 \\ 
	\hline 
	analyse & [500,1000](500,500,1000,1000) (a2) & 6 \\ 
	\hline 
	Prog & fig1(0,500,1000,1500) (a3) & 3 \\ 
	\hline 
	Doc & 500 (a1) & 3 \\ 
	\hline 
	BDD & [500,1000](500,500,1000,1000) (a2) & 6 \\ 
	\hline
\end{tabular} 
\end{center}

\begin{center}
\begin{tabular}{|c|c|c|c|c|}
	\hline
	\multicolumn{5}{|c|}{\textbf{TRAPEZE}}\\
	\hline
	\hline 
	a1 & 500 & 500 & 500 & 500 \\ 
	\hline 
	a2 & 500 & 500 & 1000 & 1000 \\ 
	\hline 
	a3 & 0 & 500 & 1000 & 1500 \\ 
	\hline 
\end{tabular} 
\end{center}

\section{Le problème}

\begin{enumerate}
\item FQUERY : \\
	\begin{itemize}
	\item prédicat flou: sous ensemble flou $f(x)$
	\item une donnée \textbf{précise} $a$
	\item le résultat
	\begin{itemize}
		\item pour une donnée: $f(a)$ le degré d’appartenance
		\item Pour la requête: des combinaisons de ces degrés
	\end{itemize}		
\end{itemize}

\item FSQL : \\
	\begin{itemize}
	\item prédicat flou: sous ensemble flou $f_A(x)$
	\item une donnée \textbf{imprécise}: un sous ensemble flou $f_{A'}(y)$
	\item le résultat
	\begin{itemize}
		\item pour une donnée est un nombre tenant compte de $f_A$ et $f_{A'} f(a)$: le degré d'appartenance $\rightarrow$ comparer deux sous ensemble flous(prédicat et donnée) 
		\item Pour la requête : des combinaisons de ces degrés
	\end{itemize}		
\end{itemize}
\end{enumerate}

\begin{center}
\begin{tabular}{|c|c|c|}
	\hline 
	\textbf{Adresse} & \textbf{Age} & \textbf{Prix} \\ 
	\hline 
	\hline
	JOFFRE & 50 & (0.5;1;$+\infty$;$+\infty$) \\ 
	\hline 
	IENA & 90 & (1.1;1.1;1.2;1.2) \\ 
	\hline 
	EIFFEL & 60 & 1.8 \\ 
	\hline 
	JOFFRE & 95 & (0.8;0.9;1.5;1.6) \\ 
	\hline 
\end{tabular} 
\end{center}

\section{Les requêtes}
\subsection{Problème}

Ayant un ensemble flou $ A $ (décrivant un prédicat: $PrixEleve$), et un ensemble flou $ A' $   (décrivant une donnée: $Prix$), on vaut mesurer la confiance en  $ A' $ est un $ A $ (le prix est un prix élevé: $PrixEleve(Prix)$)

\subsection{Solution : La théorie des possibilités}
\begin{itemize}
\item La mesure de possibilité (Optimisme)\\
$ ps(X est A / Xest A')=sup\{min(f_A(x),f_{A'}(x)) / x \in E\} $
\item la mesure de nécessité (Pessimisme)\\
$ nc(X est A / X est A')=inf \{max(1-f_{A'}(x), f_A(x)) / x \in E\} $
\end{itemize}
~

\emph{\textbf{N.B.}: Le résultat du minimum de deux fonctions floues est une fonction floue.}

\begin{center}
\begin{tabular}{|c|c|c|c|c|}
	\hline 
	\textbf{Adresse} & \textbf{Age} & \textbf{Prix} & \textbf{Ps} & \textbf{Nc} \\ 
	\hline
	\hline 
	JOFFRE & 50 & (0.8;0.9;1.1;1.2) & 0.5 & 0 \\ 
	\hline 
	IENA & 90 & (1.1;1.1;1.7;1.7) & 1 & 0.5  \\ 
	\hline 
	EIFFEL & 60 & 1.8 & 1 & 1  \\ 
	\hline 
	JOFFRE & 95 & (1.3;1.4;1.5;1.6) & 1 & 1  \\ 
	\hline 
\end{tabular} 
\end{center}
avec $f_{prixEleve}:(1;1.2;1.8;2)$

\begin{lstlisting}[mathescape]
SELECT col1,...,coln,cdeg(ps|n)
FROM T1,...,Tm
WHERE condition
\end{lstlisting}
~

\noindent \texttt{cdeg(ps|n)} rend trié sur degré de ps ou nc\\
\texttt{condition:faible(indemnité)...prixEleve(Prix)...faible(Indemnité) $>$ NFEQ 0,1} retourne les résultats avec un degré de nécéssité supérieur a 0,1\\
\texttt{faible(indemnité)$>$FEQ 0,1} retourne les résultats avec un degré de possibilité supérieur à 0,1

\subsection{TD1 Flou, SGBD}
\subsubsection{Exercice 1}
\paragraph{1) Relations floues} ~\\
	\begin{itemize}
	\item $f_{J\&K}(x)=min(f_J(x),f_K(x))$
	\item $f_{J\|K}(x)=max(f_J(x),f_K(x))$
	\end{itemize}

\paragraph{2) Théorie des possibilités} ~\\
\begin{itemize}
\item $Ps(f_J|f_{[18;18;28;28]})=sup\{min(f_J|f_{[18;18;28;28]})\}=1$
\item $Nc(f_J|f_{[18;18;28;28]})=inf\{max(1-f_{[18;18;28;28]},f_J)\}=0$
\end{itemize}
~

\emph{\textbf{N.B.}: Attention à l'ordre prédicat/données}\\

\paragraph{3) Modérateurs linguistique} ~\\

$f_{tres}(x)=x^2 $

\subsubsection{Exercice 2}

\paragraph{1) Définitions de fonction floues} ~\\

\noindent $f_{GrandDom}=[0;150;200;+\infty]$\\
$f_{Difficile}=[20;25;+\infty;+\infty]$\\
$f_{Facile}=[0;0;20;25]$\\
$f_{Moyen}=[260;280;320;340]$\\
$f_{MoyenneMontagne}=[1300;1500;1500;1700]$\\

\paragraph{2) Requetes et résultats}

\subparagraph{a) Stations avec un grand domaine skiable} ~\\
\begin{lstlisting}[mathescape]
select S.Nom
from Stations S
where $f_{GrandDom}$(S.Domaine)
\end{lstlisting}
~

\begin{tabular}{|c|c|}
	\hline 
	\textbf{Nom} & $f_{GrandDom}$ \\ 
	\hline 
	\hline
	Courch & 1 \\ 
	\hline 
	DeuxA & 0.8 \\ 
	\hline 
	Samo & 0.4 \\ 
	\hline 
	Serre & 1 \\ 
	\hline 
\end{tabular}

\subparagraph{b) Stations avec un grand domaine skiable et beaucoup de pistes difficiles} ~\\
\begin{lstlisting}[mathescape]
SELECT S.Nom
FROM Stations S
WHERE $f_{GrandDom}$(S.Domaine) & $f_{Difficile}$(S.PisteRN)
\end{lstlisting}
~

\begin{tabular}{|c|c|c|c|}
	\hline 
	\textbf{Nom} & $f_{GrandDom}$ & $f_{Difficile}$ & \textbf{Résultat} \\ 
	\hline
	\hline
	Courch & 1 & 1 & 1 \\ 
	\hline 
	DeuxA & 0.8 & 1 & 0.8 \\ 
	\hline 
	Samo & 0.4 & 1 & 0.4 \\ 
	\hline 
	Serre & 1 & 1 & 1 \\ 
	\hline 
\end{tabular}
~\\

\subparagraph{c) Stations à environ 1500m avec beaucoup de piste difficiles} ~\\

\begin{lstlisting}[mathescape]
SELECT S.Nom
FROM Stations S
WHERE $f_{MoyenneMontagne}$(S.Altitude) & $f_{Difficile}$(S.PisteRN)
\end{lstlisting}
~

\begin{tabular}{|c|c|c|c|}
	\hline 
	\textbf{Nom} & $f_{MoyenneMontagne}$ & $f_{Difficile}$ & \textbf{Résultat} \\ 
	\hline 
	\hline
	Courch & 1 & 1 & 1 \\ 
	\hline 
	Serre & 1 & 1 & 1 \\ 
	\hline 
\end{tabular}
~

\subparagraph{d) Stations ou l'on peut trouver un hotel à prix moyen} ~\\

\begin{lstlisting}[mathescape]
SELECT S.Nom
FROM Stations S natural join Hotels H
WHERE $f_{Moyen}$(H.Prix)
\end{lstlisting}
~

\begin{tabular}{|c|c|}
	\hline 
	\textbf{Nom} & $f_{Moyen}$ \\ 
	\hline 
	\hline
	Samo & 0.5 \\ 
	\hline 
	Courch & 1 \\ 
	\hline 
	Pierre & 1 \\ 
	\hline 
\end{tabular}
~\\

\emph{\textbf{N.B.}: distinct ajoute une borne sup sur la condition}

\subparagraph{e) Stations avec un hôtel à prix moyen ou un grand domaine skiable} ~\\

\begin{lstlisting}[mathescape]
SELECT S.Nom
FROM Stations S natural join Hotels H
WHERE $f_{Moyen}$(Prix) $\|$ $f_{GrandDom}$
\end{lstlisting}
~

\begin{tabular}{|c|c|c|c|}
\hline 
	\textbf{Nom} & $f_{Moyen}$ & $f_{GrandDom}$ & $f_{Moyen \| GrandDom}=max(f_{Moyen},f_{GrandDom})$ \\ 
	\hline 
	\hline
	Samo & 0.5 & 0.4 & 0.5 \\
	\hline 
	Courch & 1 & 1 & 1 \\ 
	\hline 
	Pierre & 1 & 0 & 1 \\ 
	\hline 
	DeuxA & 0 & 0.8 & 0.8 \\ 
	\hline 
	Serre & 0 & 1 & 1 \\ 
	\hline 
\end{tabular}
~

\subparagraph{f) Stations où le prix du forfait est environ la moitié du prix d'une chambre: environ, à définir, permet une tolérance linéaire de 50} ~\\

\begin{lstlisting}[mathescape]
SELECT S.Nom 
FROM Stations S natural join Hotels H
WHERE S.Forfait $\approx$ H.Prix/2
\end{lstlisting}
~
avec $f_{\approx} = |X-Y| \leq 50 $\\

\begin{tabular}{|c|c|c|c|c|c|}
	\hline 
	\textbf{Nom} & \textbf{IdStation} & \textbf{Prix/2} & \textbf{Forfait} & $f_{\approx}$ & \textbf{DISTINCT}\\ 
	\hline 
	\hline
	Samo & 102 & 125 & 150 & $25\rightarrow0.5$ & NON \\ 
	\hline 
	Samo & 102 & 165 & 150 & $15\rightarrow0.7$ & OUI \\ 
	\hline 
	Samo & 102 & 120 & 150 & $30\rightarrow0.4$ & NON \\ 
	\hline 
	Samo & 102 & 130 & 150 & $20\rightarrow0.6$ & NON \\ 
	\hline 
	Courch & 100 & 155 & 200 & $45\rightarrow0.1$ & NON \\ 
	\hline 
	Courch & 100 & 225 & 200 & $25\rightarrow0.5$ & OUI \\ 
	\hline 
	Pierre & 201 & 135 & 90 & $45\rightarrow0.1$ & OUI \\ 
	\hline 
	Pierre & 201 & 150 & 90 & $60\rightarrow 0$ & NON \\ 
	\hline 
\end{tabular} 
~

\subsubsection{Exercice 3}

\emph{\textbf{N.B}.: Lors d'un calcul de possibilitée ou necessité, il faut considerer les constantes comme des valeurs floues.}

\paragraph{1)} Il faut définir la table TRAPEZE qui contiendra les fonctions floues que l'on souhaite modéliser: Mûr, Jeune, Vieux, Elevé, Faible

\paragraph{2)} ~
\begin{lstlisting}[mathescape]
SELECT C.Nom
FROM Cadre C
WHERE $f_{moyenSup}(C.Salaire)$
\end{lstlisting}

\part{Data Ware House}
\section{Entrepot de données}
\subsection{Le problème}
	Les bases de données actuellement utilisées sont essentiellement des BD de production de type OLTP \emph{(On Line Transaction Processing)}, elles se caractérisent par:
	\begin{itemize}
		\item les mises à jour de données 
		\item un accès en lecture/écriture
		\item des échanges utilisateurs/BD faibles \emph{(échanges sur un tuple)}
		\item une orientation des interfaces vers 1D ou 2D
		\item une taille en Giga/Tera octects
		\item des données récentes
	\end{itemize}
	~\\

\emph{\textbf{Remarque:} Si on regarde la taille des Bases de données, il n'y a pas tant de données que ça (en terme de volume), surtout par rapport à la technologie actuelle. Raison ? Pas de graphiques, de vidéos, mais aussi parce que l'on travaille uniquement sur des données récentes (i.e. La SNCF n'a aucune raison de garder l'information sur qui a acheté des places pour le train qui est parti hier...).}	\\
		
		Le modèle OLAP est le modèle dominant d'aujourd'hui. \\
		
		Les bases de données d'aide à la décision émergent, elles sont de type OLAP \emph{(On Line Analysis Processing)}. L'idée est d'analyser les données de l'entreprise afin d'en faire quelque chose, par exemple pour une campagne markéting ou encore pour gérer le service finance...\\ 
		
		Elles sont caractérisées par:
		\begin{itemize}
			\item l'analyse de données \emph{(i.e. analyser les données d'un client pour savoir si on lui accorde ou non un crédit)}
			\item un accès uniquement en lecture \emph{(les données sont analysées mais ne sont pas produites donc pas besoin d'écriture)}
			\item des échanges utilisateur/BD forts \emph{(i.e. Si l'on regarde chez un fournisseur quelles sont les sommes de produits achetés l'année dernière, les échanges vont être énormes, beaucoup de données seront à récupérer dans la base de données)}
			\item une orientation des interfaces vers du 3 à n Dimensions \emph{(i.e. Comme on a plein de données à analyser soi-même, on a besoin d'interfaces complexes, qui permettent d'avoir plusieurs écrans en même temps, de naviguer plus facilement...)}
			\item une taille en Tera/Peta octets \emph{(les données sont surtout de type historique, accès des données sur le long terme, recherche sur les 10 dernières années, etc)}
		\end{itemize}
		~\\
		
		Ces BDD permettent de contenir et synthétiser les données de l'entreprise en vue de comprendre son fonctionnement.\\
		
		Traditionnellement, les BDD en entreprise étaient sur le modèle OLTP mais s'amènent petit à petit vers le modèle OLAP. Le marché du décisionnel est un marché en forte croissance, 
		
		\subsection{Le Dataware House}
		La base de données de l'entreprise est conçu par domaine d'activité de l'entreprise \emph{(personnel, stock, client, etc.)}. \\
		
		L'idée de l’Entrepôt de données (DataWare House) est de regrouper l'ensemble des données avec historique afin de faire de l'aide à la prise de décision.\\
		
		A partir de l’Entrepôt, on peut:
		\begin{itemize}
			\item faire de l'analyse de données (OLAP) pour extraire des données de synthèse essentiellement numériques. \emph{(Par exemple, je cherche le chiffre d'affaire que fait mon fournisseur avec une analyse par mois)}
			\item faire de la fouille de données (Data Mining, qui extrait des connaissances. \emph{(Donner des informations qu'on ne pensait pas forcément intéressante))}
		\end{itemize}
		
		\subsection{Une architecture}
		\begin{enumerate}
			\item Extraction (BD) \emph{(Moniteur, Adaptateur)}
			\item Fusion (DataWare House) \emph{(Médiateur, Entrepot)}
			\item Exploitation (OLAP, DataMining) \emph{(Décideur)}
		\end{enumerate}
			
	\section{La modélisation multidimensionnelle (OLAP)}
		\subsection{Le Cube de Données}
		Un cube de données est une représentation visuelle en 3D de données, il est composé de 3 axes représentant des attributs Dimi, et d'un indicateur IndX qui est la variable analysée dans le cube. La variable analysée peut être une agrégation.\\
		
		\emph{\textbf{Remarque:} l’Entrepôt de données est une base de données, comment sont stockés les bases de données ? Ça peut être un modèle relationnel, ou autre... la manière dont est stockée la BDD mériterait un cours en soit. Ici, on parlera de la manière d'exploiter la BDD avec comme exemple un entrepôt représenté en une seule et grande table.}\\
		
		Soit l'exemple ci-dessous: Achat(NomFournisseur, NomProduit, Date, Quantité, Prix)\\
		
		On va avoir le cube de données suivant en choisissant les axes \emph{Date}, \emph{Produit} et \emph{Fournisseur}:\\
		
		\begin{center}
		\begin{tikzpicture}[every node/.style={minimum size=1cm},on grid]
			\begin{scope}[every node/.append style={yslant=-0.5},yslant=-0.5]
  				\node at (0.5,2.5) {15};
  				\node at (1.5,2.5) {};
  				\node at (2.5,2.5) {};
  				\node at (0.5,1.5) {};
  				\node at (1.5,1.5) {50};
  				\node at (2.5,1.5) {};
  				\node at (0.5,0.5) {};
  				\node at (1.5,0.5) {5};
  				\node at (2.5,0.5) {10};
  				\draw (0,0) grid (3,3);
  				\draw [thick,->] (0, 0) -- (0, 3.5);
  				\node at (-0.5,2.5) {2012};
  				\node at (-0.5,1.5) {2011};
  				\node at (-0.5,0.5) {2010};
  				
  				\node[rotate=90] at (-0.5,3.5) {(Date)};
  				
  				\node at (0.5,-0.5) {P1};
  				\node at (1.5,-0.5) {P2};
  				\node at (2.5,-0.5) {P3};
  				
  				\node at (-1,-0.5) {(Produit)};
			\end{scope}
			\begin{scope}[every node/.append style={yslant=0.5},yslant=0.5]
  				\node at (3.5,-0.5) {};
  				\node at (4.5,-0.5) {};
  				\node at (3.5,-1.5) {};
  				\node at (4.5,-1.5) {};
  				\node at (3.5,-2.5) {};
  				\node at (4.5,-2.5) {18};
  				\draw (3,-3) grid (5,0);
  				\draw [thick,->] (3, -3) -- (-0.5, 0.5);
  				\node at (3.5,-3.5) {F1};
  				\node at (4.5,-3.5) {F2};
  				
  				\node at (6.5,-3.5) {(Fournisseur)};
  				
			\end{scope}
			\begin{scope}[every node/.append style={yslant=0.5,xslant=-1},yslant=0.5,xslant=-1]
  				\node at (3.5,2.5) {};
  				\node at (3.5,1.5) {};
  				\node at (3.5,0.5) {};
  				\node at (4.5,2.5) {};
  				\node at (4.5,1.5) {};
  				\node at (4.5,0.5) {};
  				\draw (3,0) grid (5,3);
  				\draw [thick,->] (0, -3) -- (2.5, -3);
			\end{scope}
			
		\end{tikzpicture}
		\end{center}
		Et on peut en sortir des vues telles que:\\
		
		Vue \textbf{F2}\\
		\begin{center}
		\begin{tikzpicture}[yscale=1, xscale=1] 
			% 3x3 grid
			\draw [color=gray] (0, 0) grid (3, 3);
			
			% x-axis label
			\node at (2.5, -0.5) {P3};
			\node at (1.5, -0.5) {P2};
			\node at (0.5, -0.5) {P1};
			% y-axis label
			\node at (-0.5, 2.5) {2012};
			\node at (-0.5, 1.5) {2011};
			\node at (-0.5, 0.5) {2010};
			
			%values
			\node at (1.5, 2.5) {15};
			\node at (0.5, 1.5) {20};
			\node at (1.5, 1.5) {7};
			\node at (1.5, 0.5) {10};
			\node at (2.5, 0.5) {18};
			
		\end{tikzpicture} 
		\end{center}
		
		Vue \textbf{2012}\\
		\begin{center}
		\begin{tikzpicture}[yscale=1, xscale=1] 
			% 3x3 grid
			\draw [color=gray] (0, 0) grid (3, 2);
			
			% x-axis label
			\node at (2.5, -0.5) {P3};
			\node at (1.5, -0.5) {P2};
			\node at (0.5, -0.5) {P1};
			% y-axis label
			\node at (-0.5, 1.5) {F1};
			\node at (-0.5, 0.5) {F2};
			
			%values
			\node at (0.5, 1.5) {15};
			\node at (1.5, 0.5) {15};
			
		\end{tikzpicture} 
		\end{center}
		
		ou encore Vue \textbf{P2, 2011}\\
		\begin{center}
		\begin{tikzpicture}[yscale=1, xscale=1] 
			% 3x3 grid
			\draw [color=gray] (0, 0) grid (2, 1);
			
			% x-axis label
			\node at (1.5, -0.5) {F2};
			\node at (0.5, -0.5) {F1};
			
			%values
			\node at (1.5, 0.5) {7};
			\node at (0.5, 0.5) {50};
			
		\end{tikzpicture} 
		\end{center}
		\subsection{L'exploitation multidimensionnelle}
		\paragraph{Granularité}
			\begin{itemize}
				\item Chaque dimension peut être granulaire et ainsi plus ou moins développée.
			\end{itemize}
		
		\paragraph{L'exploitation}
			\begin{itemize}
				\item Le \emph{dépliage (Drilldown)} est l'extension d'une dimension vers une dimension à grain plus fin
				\item Le \emph{pliage (Rollup)} est la réduction d'une dimension vers une dimension à grain plus gros
				\item La \emph{coupe (Slice)} est la sélection de tranches du cube selon une dimension
				\end{itemize}
				~\\
				
		\section{Conclusion}
		
		Le DataWare House est un domaine qui est en train de monté, avec une interface (le cube de données) qui devient un standard.
				
\part{NoSql}
	\section{Idée}			
		Le \og{}Not Only SQL\fg{} désigne une catégorie de \textsc{SGBD} qui n'est pas fondé sur le modèle relationnel. Il est privilégié pour l'utilisation de grandes quantités de données, comme par exemple pour des sites web (Google, Amazon ...). A la différence du SQL, le NoSql n'utilise pas le système de table. Il n'y a donc pas de schémas prédéfinis
	~\\

	\section{Les Modèles}
		\subsection{Représentation Clé/Valeur}
		Cette représenatation est similaire à un \og{}map\fg{} en C++. Elle permet d'optimiser la gestion des BDD sous forme de tableau associatif. Les bases de données clé-valeurs sont couramment utilisées pour stocker des données de sessions par exemple. On peut notamment citer Amazon qui utilise la solution « Amazon DynamoDB » pour leur bases de données clé-valeur, qui leurs permettent de stocker les paniers des utilisateurs de leur boutique en ligne.
		
		\paragraph{Les plus}

		Un avantage des bases de données clé-valeur est qu'elles sont très faciles et rapides à réaliser. De plus elles ont le mérite d’être facilement extensibles.	De plus, les applications utilisant une persistance avec des bases de type clé-valeur ont la particularité d'avoir un code simplifié et facile à lire par rapport à des applications standards utilisant SQL. Enfin, les bases de données clé-valeurs ont de très bonnes performances et ont tendance à réduire le nombre total de requêtes effectuées sur la base puisque les seuls types de requêtes possibles sont la 
lecture et l'écriture de données.
		
		\paragraph{Les moins}


		Du fait de l’extrême simplicité du principe des bases clé-valeur, on ne peut faire que des requêtes basées sur les clés.
		
		\subsection{Représentation orientées colonnes}
		Une base de données colonnes est tout simplement une base de données dont les données sont stockées par colonne, et nonstockées par colonne, et non par ligne. Un SGBD relationnel est organisé en lignes. Une base de données orientée colonnes n’a pas de schéma conceptuel, et peut évoluer avec le temps en nombre de colonnes tout comme en nombre de lignes.\\
		\begin{table}[h]
			\begin{center}
				\begin{tabular}{|c|c|c|c|}
				\hline 
				\textbf{id} & \textbf{Nom} & \textbf{Ecole} & \textbf{CouvreChef} \\ 
				\hline 
				\hline
				1 & Jean & & Chapeau \\ 
				\hline 
				2 & Paul & Polytech & \\ 
				\hline 
				3 & Jacques & &  \\ 
				\hline 
				4 & & & Casquette \\ 
				\hline 
				5 & & & Sombrero \\ 
				\hline	
				\end{tabular}
			\end{center}
			\caption{données d’une base de données relationnelle}
		\end{table}
	
		\begin{table}[h]
			\begin{center}
				\begin{tabular}{|c|c|c|}
				\hline 
				\textbf{Nom} & \textbf{Ecole} & \textbf{CouvreChef} \\ 
				\hline 
				\hline
				Jean (1) & Polytech (2) & Chapeau (1) \\ 
				\hline 
				Paul (2) &  & Casquette (4)\\ 
				\hline 
				Jacques (3) &  & Sombrero (5) \\ 
				\hline 	
				\end{tabular}
			\end{center}
			\caption{données d’une base de données relationnelle}
		\end{table}

	~\\
		\paragraph{Les plus}
		Dans le cas d’une base de données colonnes, les données sont compressées le plus possible. En effet, si chaque colonne contient des données qui se ressemblent beaucoup, la compression de ces données sera très significative.

		Pour le cas d’une base de données où l’on réalise plus de lectures que d’écritures et dont on veut récupérer de gros volumes d’une donnée spécifique donnée (par exemple l’évolution d’un indicateur relevé sur un sous réseau télécom durant une période définie), il est beaucoup plus performant de 
récupérer directement les informations de toute une colonne, plutôt que de parcourir toutes les lignes pour ne prendre que cette information, comme vu précédemment.

		\paragraph{Les moins}
		L’utilisation de bases de données colonnes n’est vraiment efficace et profitable dans le cas de grands volumes de données de mêmes types (une colonne pour chaque type), et dont les données au sens d’un type se ressemblent.

		\subsection{Représentation en graphes}
		Elle est basé sur les graphes en mathématiques. Une base de données orientée graphe stocke donc des données sur chaque n\oe{}ud, et ces nœuds eux-mêmes sont organisés par des relations. Il est ensuite possible d'effectuer des opérations particulières à des graphes sans avoir besoin de passer par des jointures et des requête très complexes et lourdes. Ces opérations comprennent par exemple :\\

		\begin{itemize}
			\item De parcourir le graphe à partir d'un n\oe{}ud, sur une profondeur donnée,
			\item De récupérer les relations existantes entre des nœuds ayant des propriétés similaires,
			\item De retrouver des \og{}ancêtre\fg{} communs à plusieurs n\oe{}uds,
			\item De calculer la profondeur et la largeur d'un morceau du graphe.
		\end{itemize}

		\paragraph{Les plus}
		Parfaitement adaptées à la gestion de données relationnelles, même lorsque ces données atteignent une taille très importante, les bases de données orientées graphe sont une solution ayant l’avantage d’être basée sur le modèle mathématique des graphes. Cela permet d’y appliquer des algorithmes déjà existants et optimisés, faisant de ce type de base une architecture relativement aisée à comprendre et à prendre en main.

		\paragraph{Les moins}
		Cependant, cette architecture est très spécifique aux graphes, et la formation n\oe{}ud/lien/propriété est très limitée, pour ne pas dire inadaptée, dans des cas plus classiques. Loin de remplacer les bases de données relationnelles, de même que d’autres types de bases NoSQL, les bases orientées graphe ne sont donc qu’une solution particulière à une forme particulière de BigData, et non une solution \og{}miracle\fg{} pouvant répondre à tous les besoins de ce domaine.
\end{document}
